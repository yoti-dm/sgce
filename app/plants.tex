\chapter*{Appendix A: Plants}

\subsubsection{Acid moss}
\label{acid_moss}

Dark green to black moss that grows in acid swamps.

\vspace{5mm}

\textbf{Usages:}

\begin{itemize}[noitemsep]
\item[] Herbalism: \hyperref[acid_vial]{Acid (vial)}
\item[] Poisons: \poison
\end{itemize}

\subsubsection{Addison's Blood}
\label{addisons_blood}

Bright red mushroom that grows in many places on material plane as well as higher planes.

\begin{itemize}[noitemsep]
\item[] Herbalism: \hyperref[healing_potion]{Healing potion}
\end{itemize}

\subsubsection{Angel wing}
\label{Angel wing}

White mushroom that grows from rocks and trees in many enviroments.

\vspace{5mm}

\textbf{Usages:}

\begin{itemize}[noitemsep]
\item[] Poisons: \poison
\end{itemize}

\subsubsection{Arkas grass tuft}
\label{Arkas grass tuft}

Long grass found in most grasslands.

\vspace{5mm}

\textbf{Usages:}

\begin{itemize}[noitemsep]
\item[] As is: Given to herbivore animal it grants +2 to animal handling checks against that animal for the day.
\end{itemize}

\subsubsection{Ash willow}
\label{Ash willow}

Dark red tree native to the elemental plane of fire that can grow to 120 ft tall. It thrives in heat, often growing from pools of lava. They continuously rain ash around them.They create areas of calm, slow-bunring forests within the volatile plane.

\vspace{5mm}

\textbf{Usages:}

%\begin{itemize}[noitemsep]
%\end{itemize}

\subsubsection{Atramen}
\label{Atramen}

A fruit that grows on shoals of the elemental plane of earth

\vspace{5mm}

\textbf{Usages:}

\begin{itemize}[noitemsep]
\item[] Herbalism: \hyperref[Atramen oil]{Atramen oil}
\item[] Poisons: \poison
\end{itemize}

\subsubsection{Baited Breath}
\label{baited_breath}

A sweet smelling bundle of buds, commonly eaten by small forest animals.

\subsubsection{Bated Breath}
\label{bated_breath}

Copy-cat carnivorous plant that causes asphyxiation in the small animals which eat it, which it then uses as fertilizer to grow.

Infant plants are easily told apart by whether or not they are growing out of a corpse.

\subsubsection{Blade grass}
\label{Blade grass}

Tall grass that grows in rivers, ponds and swamps. Slightly poisonous.

\vspace{5mm}

\textbf{Usages:}

\begin{itemize}[noitemsep]
\item[] Poisons: \poison
\end{itemize}

\subsubsection{Barrelstalk}
\label{Barrelstalk}

Large cask-shaped fungus that grows to 5 x 8 ft. Outside is hard as wood, inner flesh is edible, center contains 20 to 50 gallons of water. After 10 years of growth, it beings spore production and the flesh grows black and poisonous.

\vspace{5mm}

\textbf{Usages:}

\begin{itemize}[noitemsep]
\item[] As is: When young it is a good source of water
\item[] Poisons: \poison
\end{itemize}

\subsubsection{Bloodroot}
\label{bloodroot}

This unassuming looking leafy vine grows from a single source. When pulled up, which can be difficult, a large red root can be seen. Dangerous by itself, but with some certain reagents it can be a powerful antitoxin.

\subsubsection{Bluecap}
\label{Bluecap}

Common underdark crop. Fugus itself is indedible, but spores can be made into flour.

\vspace{5mm}

\textbf{Usages:}

\begin{itemize}[noitemsep]
\item[] Herbalism: \hyperref[Bluecap powder]{Bluecap powder}
\end{itemize}

\subsubsection{Blueleaf}
\label{Blueleaf}

Durable tree with gleaming blue leaves native to arctic and subarctic latitudes. Reaches 40 ft height, rarely develops thick trunks, bends rather than breaks under wind/snow and tends to grow in thick strands.

\vspace{5mm}

\textbf{Usages:}

\begin{itemize}[noitemsep]
\item[] As is: Wood burns with a blue flame
\item[] Carpentry: Prime material
\item[] Herbalism: \hyperref[Blueleaf dye]{Blueleaf dye}
\end{itemize}

\subsubsection{Bog Myrtle}
\label{bog_myrtle}

Round green leaves, and with flowering cones that look like berries on them. Used to flavor beer sometimes, but can be crushed, burned, and inhaled to cause psychotropic effects, and to be used as a painkiller. Afterwards, it is known to cause some hard migraine headaches. Used often by Berserkers, giving it the nickname Berserker Weed.

\subsubsection{Bone fungus}
\label{Bone fungus}

Bulbous ivory-coloured fungus that reseales a 10 ft cloud of spores that work as an inhaled poison (1d6 poison damage, DC 14 DEX saving throw reduces to half). Found in caves.

\vspace{5mm}

\textbf{Usages:}

\begin{itemize}[noitemsep]
\item[] Poisons: \poison
\end{itemize}

\subsubsection{Boomsrhoom}
\label{Boomsrhoom}

Special environmental conditions in swamps and marshes can give rise to up to 300 ft radius patches of these magical mushrooms. When disturbed, boomshorms explode dealing 1d4 fire damage in radius of 5 feet (DC 15 DEX saving throw to avoid). Tools in herbalism kit or DC 20 survival check allow you to pick one, and it renders it inert. Always regrow in the same spot in 10 days after triggering.

\vspace{5mm}

\textbf{Usages:}

\begin{itemize}[noitemsep]
\item[] Herbalism: \hyperref[Red powder]{Red powder}
\end{itemize}

\subsubsection{Bronzewood}
\label{Bronzewood}

Hard wood native to lands of Irian and Lakland that is as durable as steel but even lighter. It can't be used for chain weapons and armor though as it can't be molded.

\vspace{5mm}

\textbf{Usages:}

\begin{itemize}[noitemsep]
\item[] Carpentry: Prime material
\end{itemize}

\subsubsection{Cassil}
\label{Cassil}

Mustard-like shrub that grows in mountains at high altitudes.

\vspace{5mm}

\textbf{Usages:}

\begin{itemize}[noitemsep]
\item[] Herbalism: \hyperref[Cassil bitter tea]{Cassil bitter tea}
\item[] Poisons: \poison
\end{itemize}

\subsubsection{Cave moss}
\label{Cave moss}

A spiecies of moss that is grazed by giant vermin and rothe. Grows in colder climates and underground.

\subsubsection{Cave star}
\label{Cave star}

Glowing yellow lichen found on chill cave roofs, that can't stand warm places.

\vspace{5mm}

\textbf{Usages:}

\begin{itemize}[noitemsep]
\item[] Herbalism: \hyperref[Cave star lantern]{Cave star lantern}
\end{itemize}

\subsubsection{Choke Mold}
\label{Choke Mold}

Bright yellow mold native to the elemental plane of earth commonly found in patches that devour air, creating localized vacuums (can cause air-tight space to collapse inwards). Creatures within 5 ft of a patch begin to suffocate.

\vspace{5mm}

\textbf{Usages:}

\begin{itemize}[noitemsep]
\item[] Herbalism: \hyperref[Choke mold powder]{Choke mold powder}
\end{itemize}

\subsubsection{Cinderpetals}
\label{Cinderpetals}

These bright red and orange flowers can be found all around the world that grow all alone. If ground up finely and left in a vial or bottle they will coagulate the water into a thick goo and make the color into a vibrant swirl of red and orange. These are commonly used to make alchemist's fire.

\subsubsection{Coldwood}
\label{Coldwood}

A Fey-created variety of Hickory that reaches maturity in 2 decades and has properties identical to steel, except the fey/druid-aversion part.

\vspace{5mm}

\textbf{Usages:}

\begin{itemize}[noitemsep]
\item[] Carpentry: Prime material
\end{itemize}

\subsubsection{Cotsbalm}
\label{Cotsbalm}

Fleshy-leafed plant with small yellow flowers that grows 1 ft tall. It is hardy and found in temperate and sub-tropical regions.

\vspace{5mm}

\textbf{Usages:}

\begin{itemize}[noitemsep]
\item[] Herbalism: \hyperref[Costbalm syrup]{Costbalm syrup}
\end{itemize}

\subsubsection{Covadish}
\label{Covadish}

Plant that grows in deep forests and feeds off of a dead plant matter.

\vspace{5mm}

\textbf{Usages:}

\begin{itemize}[noitemsep]
\item[] Herbalism: \hyperref[Covandish knot]{Covandish knot}
\end{itemize}

\subsubsection{Darkberry}
\label{Darkberry}

Bush that grows clumps of small purple berries. Found on graslands and in forests.

\vspace{5mm}

\textbf{Usages:}

\begin{itemize}[noitemsep]
\item[] Herbalism: \hyperref[Exploding darkberry]{Exploding darkberry}
\end{itemize}

\subsubsection{Darkroot}
\label{Darkroot}

Large black twisted rout that grows to 10 ft and is found by waterfalls and similar damp areas. It tastes vile, if tasted or swallowed it induces vomiting.

\vspace{5mm}

\textbf{Usages:}

\begin{itemize}[noitemsep]
\item[] Herbalism: \hyperref[Titan gum]{Titan gum}
\item[] Poisons: \poison
\end{itemize}

\subsubsection{Darkshine}
\label{Darkshine}

Crystallizing glossy-black fungus hat grows sharp 6-ft long shards, native to elemental plane earth. Areas with Darkshine in it function as clatrops that deal 1d10 damage, and damaged creature can not be magically healed for 1 minute.

\vspace{5mm}

\textbf{Usages:}

\begin{itemize}[noitemsep]
\item[] Poisons: \poison\poison
\end{itemize}

\subsubsection{Darkwood aka. Zalantar}

Wood items made from this tree's magical wood are half-weight but a bit weaker than their iron counterparts.

\vspace{5mm}

\textbf{Usages:}

\begin{itemize}[noitemsep]
\item[] Carpentry: Prime material
\end{itemize}

\subsubsection{Deep Imaskari Waterplant}
\label{Deep Imaskari Waterplant}

These magical plants are found in deep deserts and grow 2-inch diameter balls of water every day that can be carried like oranges, until they are pierced and drank from.

\vspace{5mm}

\textbf{Usages:}

\begin{itemize}[noitemsep]
\item[] As is: Water and water transport
\end{itemize}

\subsubsection{Densewood}

Hard heavy and strudy tree native to Irian that can be used in armor and heavy weapons making.

\vspace{5mm}

\textbf{Usages:}

\begin{itemize}[noitemsep]
\item[] Carpentry: Prime material
\end{itemize}

\subsubsection{Devil Weed}

Tobaccolike weed that can be smoked in pipes or as a cigar.

\subsubsection{Djin Blossom}
\label{Djin Blossom}

Rare fern native to the Elemental Plane of Air.

\vspace{5mm}

\textbf{Usages:}

\begin{itemize}[noitemsep]
\item[] Herbalism: \hyperref[Djin blossom roll]{Djin blossom roll}
\end{itemize}

\subsubsection{Dragon's Breath}
\label{dragons_breath}

A hard lichen found growing on stone found in mountain caverns that when crushed produces an oily ichor that is very volatile, often creating small but violent explosions on procuring. It's main purpose is for creating a substance akin to black powder by mixing it with sulfur. It also has many other interesting effects.

When it comes into contact with metal, it causes it to appear as if it were gold for a brief period of time. After the effect wears off a thin film of rust is left covering the metal. By combining the ichor with a solvent, the color changing property is lost, while the ability to rust metal is greatly amplified

When combined with water, the ichor causes the liquid to faintly glow producing light equivalent to a torch. Examination of the liquid provides a stunning display, appearing as if slow flickering flames are suspended in the water which eventually evaporates. When evaporated, an inert powder is left behind which little is known about. The effects of using the liquid in ways other than a light-source are unknown, as gathering knowledge into the magical nature of the concoction has proved fruitless. and no one has been foolish enough to attempt to imbibe the mysterious concoction

\subsubsection{Dragonseye Oak}
\label{Dragonseye Oak}

Rare species of oak that grows in an open bright forests and fields.

\vspace{5mm}

\textbf{Usages:}

\begin{itemize}[noitemsep]
\item[] Herbalism: \hyperref[Dragonseye oak accorn shell]{Dragonseye oak accorn shell}
\end{itemize}

\subsubsection{Dwarven Oak}
\label{Dwarven Oak}

Stunted, gnarled tree found on the slopes of temperate mountains that looks like a sitting Dwarf from a distance.

\vspace{5mm}

\textbf{Usages:}

\begin{itemize}[noitemsep]
\item[] Herbalism: \hyperref[Dwarven oak syrup]{Dwarven oak syrup}
\item[] Poisons: \poison
\end{itemize}

\subsubsection{Duskwood}

Black-barked tree that grows 60 ft tall in tightly spaces groves, featuring small branches and smokey grey wood that's strong as iron  but half the weight.

\vspace{5mm}

\textbf{Usages:}

\begin{itemize}[noitemsep]
\item[] Carpentry: Prime material
\end{itemize}


\subsubsection{Eldritch Whorlwood}

Tree with a twisted and gnarled grain pattern, which becomes straight if a wand or other charged magic item made of eldritch whorlwood expends all its charges.

\subsubsection{Elven Willow}
\label{Elven Willow}

Small tree (up to 5 ft tall) with golden-sheen bark that produces golden buds in the spring. Grows on riverbanks in temperate areas.

\vspace{5mm}

\textbf{Usages:}

\begin{itemize}[noitemsep]
\item[] Herbalism: \hyperref[Elf Hazel]{Elf Hazel}
\end{itemize}

\subsubsection{Ember root}
\label{Ember root}

A shrivelled-coconut looking plant native to the elemental plane of fire that grows on any solid stone in areas of at least extreme heat. The flesh is poisonous, but the core contains drinkable liquid that never grows hot.

\vspace{5mm}

\textbf{Usages:}

\begin{itemize}[noitemsep]
\item[] Herbalism: \hyperref[Ember root oil]{Ember root oil}
\item[] Poisons: \poison
\end{itemize}

\subsubsection{Entangle Weed}
\label{Entangle Weed}

Fully translucent seaweed native to the elemental plane of water that forms patches of about 600 ft diameter. Creatures can get entangled into it and die.

\vspace{5mm}

\textbf{Usages:}

\begin{itemize}[noitemsep]
\item[] Herbalism: \hyperref[Translucent paint]{Translucent paint}
\end{itemize}

\subsubsection{False morel}
\label{False morel}

False morels have wrinkled, irregular caps that are brainlike or saddle-shaped. They may be black, gray, white, brown, or reddish.

\vspace{5mm}

\textbf{Usages:}

\begin{itemize}[noitemsep]
\item[] Poisons: \poison
\end{itemize}


\subsubsection{Fesul}
\label{Fesul}

A type of gnarled, twisted tree that favors cold and poor soil and areas with little purchase such as cliffsides. Its cinnamon-brown wood crumbles when touched.

\vspace{5mm}

\textbf{Usages:}

\begin{itemize}[noitemsep]
\item[] Herbalism: \hyperref[Fesul parfume]{Fesul parfume}
\end{itemize}

\subsubsection{Fey cherry}
\label{Fey cherry}

Unbelievably massive cherry trees that can live forever. Area under its canopy is mystically protected, always temperate and windspeeds are dampened by 20 mph. It blossoms annually but only creates cherries every 10 years. Eating a cherry picked less than a day ago grants a protection from evil and good for 5 minutes.

\vspace{5mm}

\textbf{Usages:}

\begin{itemize}[noitemsep]
\item[] Herbalism: \hyperref[Fey cherry tea]{Fey cherry tea}, \hyperref[Fey cherry juice]{Fey cherry juice}
\end{itemize}

\subsubsection{Fire lichen}
\label{Fire lichen}

Orange white lichen that growths in warm underground areas. Can be made into a spicy paste or fiercly hot liquor.

\vspace{5mm}

\textbf{Usages:}

\begin{itemize}[noitemsep]
\item[] Herbalism: \hyperref[Fire lichen sauce]{Fire lichen sauce}, \hyperref[Fire lichen liquor]{Fire lichen liquor}
\item[] Poisons: \poison
\end{itemize}

\subsubsection{Fire moss}
\label{fire_moss}

Found only underground, this dull red moss is extremely long burning and can be wrapped around a stick and used as a torch for approximately three hours.

\subsubsection{Flame clove}
\label{Flame clove}

Garlic with essence of the elemental plane of fire. Poisonous to most beings even more so to vampires.

\vspace{5mm}

\textbf{Usages:}

\begin{itemize}[noitemsep]
\item[] Herbalism: \hyperref[Vampire bane]{Vampire bane}
\end{itemize}

\subsubsection{Fleshshiver}
\label{Fleshshiver}

Ten-coloured mushroom that grows in the soil between the roots of tropical fruit trees.

\vspace{5mm}

\textbf{Usages:}

\begin{itemize}[noitemsep]
\item[] Herbalism: \hyperref[Fever cloth]{Fever cloth}
\end{itemize}

\subsubsection{Frog Seat}
\label{frog_seat}

A giant toadstool that has a slightly slimy texture. Can be ground into a paste and used to ease aching joints. It is sometimes used to lubricate armour which dulls the shine of steel and softens the sounds of plate moving.

\subsubsection{Galda}
\label{Galda}

A yellowish tree that produces a salty fruit.

\subsubsection{Glowvine}
\label{Glowvine}

A morning glory derivative that gives of light as a torch at night.

\vspace{5mm}

\textbf{Usages:}

\begin{itemize}[noitemsep]
\item[] Herbalism: \hyperref[Glowvine torch]{Glowvine torch}
\end{itemize}

\subsubsection{Goblin Rogue}
\label{Goblin Rogue}

Medium-sized bush with yellow-orange berries in autumn found in temperate regions.

\vspace{5mm}

\textbf{Usages:}

\begin{itemize}[noitemsep]
\item[] Herbalism: \hyperref[Goblin ink]{Goblin ink}
\end{itemize}

\subsubsection{Golden Desert Tree}
\label{Golden Desert Tree}

A rare desert tree, whose sap is a vital ingredient in expensive perfumes and incense.

\vspace{5mm}

\textbf{Usages:}

\begin{itemize}[noitemsep]
\item[] Herbalism: \hyperref[Golden desert tree incense]{Golden desert tree incense}, \hyperref[Golden desert tree parfume]{Golden desert tree parfume}
\end{itemize}

\subsubsection{Goldencup}
\label{Goldencup}

Oily yellow moss found where water collects near the bottom of rocks in tundras.

\vspace{5mm}

\textbf{Usages:}

\begin{itemize}[noitemsep]
\item[] Herbalism: \hyperref[Dried goldencup]{Dried goldencup}
\end{itemize}

\subsubsection{Green Air Bramble}
\label{Green Air Bramble}

Fast-growing creping vine that sprouts green berries. Can grow in most inhospitable climates and only needs to be in moist soil for 6 hours per week. Exposure to poison quickly kills the plant, wrinkling leaves and berries.

\vspace{5mm}

\textbf{Usages:}

\begin{itemize}[noitemsep]
\item[] Herbalism: \hyperref[Green air bramble bracelet]{Green air bramble bracelet}
\end{itemize}

\subsubsection{Green algae}

Common algae find in shallow waters.

\subsubsection{Green amanita}
\label{Green amanita}

Extremely poisonous green mushroom.

\vspace{5mm}

\textbf{Usages:}

\begin{itemize}[noitemsep]
\item[] Poisons: \poison\poison
\end{itemize}

\subsubsection{Gripweed}
\label{gripweed}

This weed is commonly found in graves and where the dead have fallen. While living it is a mass that swirls and wraps around
itself. If it gets a hold of you it tries to pull you into its mass to digest you. If someone is lucky enough to reap this it
can be made into a common poison that cuts off the victims airway

\subsubsection{Gulthias Tree}
\label{Gulthias Tree}

A severely evil tree that came to be when a vampire was staked to the ground with a stake that was still green and took root.

\vspace{5mm}

\textbf{Usages:}

\begin{itemize}[noitemsep]
\item[] Herbalism: \hyperref[Gulthias tree concentrat]{Gulthias tree concentrat}
\item[] Poisons: \poison\poison
\end{itemize}

\subsubsection{Halfling Thistle}
\label{Halfling Thistle}

Small hardy thistle with a violet flower that grows in all temperature areas, especially highlands.

\vspace{5mm}

\textbf{Usages:}

\begin{itemize}[noitemsep]
\item[] Herbalism: \hyperref[Shinewater]{Shinewater}
\end{itemize}

\subsubsection{Hanging web leaf}
\label{hanging_web_leaf}

A fine silver vine that hangs off ledges. It will often weave itself together and looks a lot like spider web but mush stronger. Often used for medical stitching because of it's gradual brake down inside the body. It can also be used to create certain kinds of antiseptic for animal bites.

\subsubsection{Harrada}
\label{Harrada}

Blood red plant native to lower plains that can defend itself by lashing out dealing 1d6 slashing damage.

\vspace{5mm}

\textbf{Usages:}

\begin{itemize}[noitemsep]
\item[] Poisons: \poison
\end{itemize}

\subsubsection{Hathil}
\label{Hathil}

Plant native to the large swamps.

\vspace{5mm}

\textbf{Usages:}

\begin{itemize}[noitemsep]
\item[] Herbalism: \hyperref[Hathil ooze]{Hathil ooze}
\end{itemize}

\subsubsection{Healing Apple Tree}
\label{Healing Apple Tree}

A magically bred medium-sized apple tree that bears red fruits that when consumed heal 1 health point.

\vspace{5mm}

\textbf{Usages:}

\begin{itemize}[noitemsep]
\item[] Herbalism: \hyperref[Infused healing apple]{Infused healing apple}
\end{itemize}

\subsubsection{Helmthorn}
\label{Helmthorn}

Very hardy and adaptable ground shrub with dark waxy green leaves and black thorns as long as human hands, which can be used as needles or dart points. Produces indigo coloured berries with a tart flavour that can be used for winemaking. Occasionally, a sring of berries will be scarlet red instead of indigo. If Goodberry is cast on a Red Helmthorn Berry, the effect lasts for one extra day.

\vspace{5mm}

\textbf{Usages:}

\begin{itemize}[noitemsep]
\item[] Herbalism: \hyperref[Helmthorn paste]{Helmthorn paste}, \hyperref[Helmthorn tea]{Helmthorn tea}
\end{itemize}

\subsubsection{Ice Lotus}
\label{Ice Lotus}

Solitary translucent blue-white flower found in cold environments.

\vspace{5mm}

\textbf{Usages:}

\begin{itemize}[noitemsep]
\item[] Herbalism: \hyperref[Icewalker oil]{Icewalker oil}
\end{itemize}

\subsubsection{Roanwood}

Trees that grow over 100 ft.

\subsubsection{Ironvine}

A type of Underdark vine that is as hard as iron. Always found interwoven into a thick curtain that blocks passage.

\subsubsection{Jabberweed}

Tenacious ugly root native to lower plains that looks like a pocked, multi-digit skeletal hand with lots of holes in it that cause a low hissing sound audible to 100 ft that imposes a penalty to other listen checks. 

\subsubsection{Kieros}
\label{Kieros}

Herb native to deep jungles and feywild.

\vspace{5mm}

\textbf{Usages:}

\begin{itemize}[noitemsep]
\item[] Herbalism: \hyperref[Kieros incense]{Kieros incense}
\end{itemize}

\subsubsection{Kiss of Discord aka. Lusiri Blossom}
\label{Kiss of Discord}

Herb with dull red leaves that resemble lips.

\vspace{5mm}

\textbf{Usages:}

\begin{itemize}[noitemsep]
\item[] Herbalism: \hyperref[Lustri blossom oil]{Lustri blossom oil}
\end{itemize}

\subsubsection{Lakeleaf}
\label{Lakeleaf}

Parsley like herb descended from plants growing on the shores of big rivers. If crushed and rubbed onto meat, that meat never dries out regardless of how overcooked.

\vspace{5mm}

\textbf{Usages:}

\begin{itemize}[noitemsep]
\item[] Herbalism: \hyperref[Lakeleaf powder]{Lakeleaf powder}
\end{itemize}

\subsubsection{Lichbriar}
\label{Lichbriar}

Bougainvillea looking plant that clings to any surface and grows up to 50 ft in ideal conditions. It has poisons thorns and can grow roots into living creatures, slowly draining their blood until they die. 

\subsubsection{Limb plant}
\label{limb_plant}

Limb seed oil: The distilled oil of the rare limb plant's seed. It can slowly re grow arms and legs. The oil is very painful and slow acting, re growing mere inches of limb every week but is far cheaper that the magical alternatives (but still not cheap). The plant is a huge bush with dark red tear drop berries with white seeds. A mature bush could have over 100 seeds but 1 vial takes around 50 seeds and two days to distil, re growing every few years if harvested properly.

\subsubsection{Lish}
\label{Lish}

Small tree that grows dozens of small nuts in the spring. A handful of the nuts sustains a medium creature for a day.

\vspace{5mm}

\textbf{Usages:}

\begin{itemize}[noitemsep]
\item[] Herbalism: \hyperref[Roasted lish nut]{Roasted lish nut}
\end{itemize}

\subsubsection{Livewood}
\label{Livewood}

Highly magical green-coloured tree native to Feywild whose wood remains alive when felled. Items made from it are affected by plant growth, sprouting small branches and leaves. Speak with plants allows one to communicate with item and blight damages them as if they were plant creatures. Livewood items can also be used for tree stride, animate plants can animate a Livewood object, and dryads can live in livewood objects. As a living object, a livewood items are immune to the disintegrate spell.

\subsubsection{Living Wood aka. Lifewood}
\label{Lifewood}

This special living wood is only found in elven forests, where it was specially bred. It heals 1 HP per round if you are touching living tree.

\vspace{5mm}

\textbf{Usages:}

\begin{itemize}[noitemsep]
\item[] Herbalism: \hyperref[Preserved lifewood thistle]{Preserved lifewood thistle}
\end{itemize}

\subsubsection{Luurden aka. Bloodfruit}
\label{Luurden}

Pale gnarled tree that looks dead, except for a short period every 3 or 4 years where it produces bitter red fruits.

\subsubsection{Maiden's Hair aka. Earthsilk}

An odd mushroom cultivated by dwarves for the silken tendrils that hang from it and collect moisture. These tendrils are tough and time-consuming to harvest, but they can be made into yarn that can create a very tough silk that make tough rope and shirts that grant protection from slashing or bludgeoning, although the shirt can be torn by a piercing damage critical hit, at which point it looses its properties until repaired.

\subsubsection{Masthin}
\label{Masthin}

Plant native to jungles that produces natural chemicals when young that attract wild animals.

\vspace{5mm}

\textbf{Usages:}

\begin{itemize}[noitemsep]
\item[] Herbalism: \hyperref[Masthin scent]{Masthin scent}
\end{itemize}

\subsubsection{Meadow Giant}
\label{Meadow Giant}

Tenacious garge green-stemmed weed that can spring up over night in temperate grasslands, plains and farmlands. It often threatens crops.

\vspace{5mm}

\textbf{Usages:}

\begin{itemize}[noitemsep]
\item[] Herbalism: \hyperref[White sanguine]{White sanguine}
\end{itemize}

\subsubsection{Mimetic Plants}

A non-specific category of plants whose fibers have the ability to take on the hue of whatever is around it.

\subsubsection{Mordayn}
\label{Mordayn}

Rare Herb found in deep forest. Has recognizable smell when crushed between fingers. Used in drug making.

\vspace{5mm}

\textbf{Usages:}

\begin{itemize}[noitemsep]
\item[] Herbalism: \hyperref[White angel]{White angel}
\item[] Poisons: \poison
\end{itemize}

\subsubsection{Mule Pollen}
\label{Mule Pollen}

A daisy-type yellow flower found in grasslands and fields.

\vspace{5mm}

\textbf{Usages:}

\begin{itemize}[noitemsep]
\item[] Herbalism: \hyperref[Tea of berserker]{Tea of berserker}
\end{itemize}

\subsubsection{Musk Muddle}
\label{Musk Muddle}

Stinky, brown, dead-looking plant with wide leaves found in swamps and marshes.

\vspace{5mm}

\textbf{Usages:}

\begin{itemize}[noitemsep]
\item[] Herbalism: \hyperref[Burn salve]{Burn salve}
\end{itemize}

\subsubsection{Nahre Lotus}
\label{Nahre Lotus}

Water lilly native to the elemental plane of water that draws water from its home plane at a rate of 50 gallons per day.

\vspace{5mm}

\textbf{Usages:}

\begin{itemize}[noitemsep]
\item[] Herbalism: \hyperref[Nahre lotus bottle]{Nahre lotus bottle}
\end{itemize}

\subsubsection{Nararoot}
\label{Nararoot}

Woody black tuber with a licorice flavor.

\vspace{5mm}

\textbf{Usages:}

\begin{itemize}[noitemsep]
\item[] Herbalism: \hyperref[Lover's tea]{Lover's tea}
\end{itemize}

\subsubsection{Nightshade}
\label{Nightshade}

Small bush that grows purple berries.

\vspace{5mm}

\textbf{Usages:}

\begin{itemize}[noitemsep]
\item[] Poisons \poison
\end{itemize}

\subsubsection{Obaddis Leaf}

Rare holly variety that can retain some magic if used as druidic focus.

\subsubsection{Old Man's Friend}
\label{Old Man's Friend}

Sticky leaf herb that grows to 2 inches in large (up to 10 by 10 ft) beds that acts like catnip for dogs.

\vspace{5mm}

\textbf{Usages:}

\begin{itemize}[noitemsep]
\item[] Herbalism: \hyperref[Gash glue]{Gash glue}
\end{itemize}

\subsubsection{Orevine}

A vine-grape looking plant often used by miners as it draws metal from the surrounding soil. Leaves are made fully from the ore it grows in.

\subsubsection{Orticusp}
\label{Orticusp}

Extremely rare flower with a root that looks like a pale white fist, found in temperate forests with trees of at least 150 years of age. Fey within 20 yards can smell its earthy aroma and find it easily.

\vspace{5mm}

\textbf{Usages:}

\begin{itemize}[noitemsep]
\item[] Herbalism \hyperref[Night venom]{Night venom}
\end{itemize}

\subsubsection{Oruighen}

A rare, very poisonous cactus native to alkaline salts.

\subsubsection{Panaeolo}
\label{Panaeolo}

Magical herb whose leaves taste like leather. Plant usually grows close to highly magical places.

\vspace{5mm}

\textbf{Usages:}

\begin{itemize}[noitemsep]
\item[] Herbalism: \hyperref[Dried panaeolo leaf]{Dried panaeolo leaf}
\end{itemize}

\subsubsection{Pixie table}
\label{Pixie table}

Rare 1-ft tall and 1 ft diameter mushroom with a dark lavender cap that can be found in any woodland but is most common in forests housing Fey. If boiled with cloth, it dyes it lavender.

\vspace{5mm}

\textbf{Usages:}

\begin{itemize}[noitemsep]
\item[] Herbalism: \hyperref[Memorybind]{Memorybind}
\end{itemize}

\subsubsection{Poison apple tree}
\label{Poison apple tree}

A magically bred medium-sized apple tree that bears red fruits that taste good but are highly poisonous.

\vspace{5mm}

\textbf{Usages:}

\begin{itemize}[noitemsep]
\item[] As is: When injested creature must succed DC 16 CON saving throw or it falls asleep until waken as an action or it receives damage.
\item[] Poisons \poison\poison
\end{itemize}

\subsubsection{Poisonous algae}
\label{Poisonous algae}

Purple algae poisonous to fish and humans alike.

\vspace{5mm}

\textbf{Usages:}

\begin{itemize}[noitemsep]
\item[] Poisons: \poison
\end{itemize}

\subsubsection{Pomow}
\label{Pomow}

A magically created dark-purple spheroid fruit-plant that grows to 1-2 feet across, serves as a hardy crop viable in a range of climates. Meat, root and seeds are edible and high in protein, core of the plant is filled with juice, the fibres are similar to cotton, and the rind can hold a razor edge. A new fruit starts growing as soon as the old one is plucked.

\vspace{5mm}

\textbf{Usages:}

\begin{itemize}[noitemsep]
\item[] Herbalism \hyperref[Pomow syrup]{Pomow syrup}
\end{itemize}

\subsubsection{Prickly Tea}
\label{Prickly Tea}

Thorny bush about 3 ft in height with grey-green leaves.

\vspace{5mm}

\textbf{Usages:}

\begin{itemize}[noitemsep]
\item[] Herbalism \hyperref[Senses]{Senses}
\end{itemize}

\subsubsection{Quasar Poppy}
\label{quasar_poppy}

A poppy with a magic based reproductive cycle, creating a colourful explosion when it wants to spread it's seeds. The poppy head can be used to create potion versions of the spell "faire fire" or the seed casings can be ground to make brightly coloured dyes that periodically change colour when used on cloth.

\subsubsection{Rare blue Mushroom}
\label{Rare blue Mushroom}

Poisonous blue mushroom that grows in forests.

\vspace{5mm}

\textbf{Usages:}

\begin{itemize}[noitemsep]
\item[] Poisons: \poison
\end{itemize}

\subsubsection{Razorvine}
\label{Razorvine}

Twinning climber native to the Lower planes that is almost impossible to get rid of as it grows at least 1 ft per day even if cut down to a stub. Dried Razorvine provides excellent fire kindle. Light contact deals 1d6 points of damage, while into it deals up to 3d6 points of damage.

\vspace{5mm}

\textbf{Usages:}

\begin{itemize}[noitemsep]
\item[] Poisons: \poison
\end{itemize}

\subsubsection{Reath}
\label{Reath}

Parasitic vine that grows on trees.

\vspace{5mm}

\textbf{Usages:}

\begin{itemize}[noitemsep]
\item[] Herbalism: \hyperref[Reath knot]{Reath knot}
\end{itemize}

\subsubsection{Red amanita}
\label{Red amanita}

Iconic red mushroom found in most forests.

\vspace{5mm}

\textbf{Usages:}

\begin{itemize}[noitemsep]
\item[] Poisons: \poison
\end{itemize}

\subsubsection{Redflower}
\label{Redflower}

Tiny red-bog flower that glows in the dark.

\vspace{5mm}

\textbf{Usages:}

\begin{itemize}[noitemsep]
\item[] Herbalism: \hyperref[Redflower powder]{Redflower powder}
\end{itemize}

\subsubsection{Red moss}
\label{Red moss}

Red moss growing in the darkest parts of old forest that feeds of decaying plant matter.

\vspace{5mm}

\textbf{Usages:}

\begin{itemize}[noitemsep]
\item[] Poisons: \poison
\end{itemize}

\subsubsection{Red podostroma}
\label{Red podostroma}

Red finger shaped fungus.

\vspace{5mm}

\textbf{Usages:}

\begin{itemize}[noitemsep]
\item[] Poisons: \poison
\end{itemize}

\subsubsection{Ripplebark}

Shelf-like fungus that looks like rooting flesh but is perfectly edible, although it tastes better if cooked properly.

\subsubsection{Ripplewood}
\label{Ripplewood}

Dark-green vine up to 400 ft long with no roots or leaves native to the elemental planes of air. Forms massive twisted nests of at least 4 vines that choose their down to be in the centre between them to float about the elemental plane of air. Often used by giant eagles for nesting. A cluster can support 500 lb per 5 ft square.

\vspace{5mm}

\textbf{Usages:}

\begin{itemize}[noitemsep]
\item[] Herbalism: \hyperref[Ripplewood tea]{Ripplewood tea}
\end{itemize}

\subsubsection{Rose of dawn}
\label{rose_of_dawn}

The petals of this rose are colored like a sunrise, and can be used to give food a spicy flavor, on the level of a ghost pepper.

\subsubsection{Ruby Apple Tree}

A legendary tree allegedly created by a by a female elf most severely gifted in the cultivation of magically grown plants. Its an apple tree that grows rubies instead of apples.

\subsubsection{Sable Fir}

A type of tree from a eponymous forest that allegedly makes excellent arrow-shafts and turns a deep lustrous black if lumbered mid-winter and rubbed with hot oils.

\subsubsection{Salamander Orchid}
\label{Salamander Orchid}

Orchid that's constantly on fire native to the elemental plane of fire that subsists on its home-planes energy wherever it is.

\vspace{5mm}

\textbf{Usages:}

\begin{itemize}[noitemsep]
\item[] Herbalism: \hyperref[Salamander's tongue]{Salamander's tongue}
\end{itemize}

\subsubsection{Sand vine}
\label{Sand vine}

Relatively rare rope-like seaweed found along temperate or warmer coasts. Grows both above and below water, commonly rooted to a small rock. Can be dried and used as rope.

\vspace{5mm}

\textbf{Usages:}

\begin{itemize}[noitemsep]
\item[] Herbalism: \hyperref[Dead pirate's elixir]{Dead pirate's elixir}
\end{itemize}

\subsubsection{Screaming Knot vine}
\label{screaming_knot_vine}

A large Knotted wine often used as rope or bow line. When stewed it creates an antiseptic brew that dulls the senses but suppresses the effects of certain painful diseases. It gets it's name from the sound released when boiled and the high pitched "twang" that comes from using it for bows.

\subsubsection{Scholar's dream}
\label{Scholar's dream}

White ivy that grows on sage graves.

\vspace{5mm}

\textbf{Usages:}

\begin{itemize}[noitemsep]
\item[] Herbalism: \hyperref[Water of future]{Water of future}
\end{itemize}

\subsubsection{Serren Wood}
\label{Serren Wood}

Tree that grows at high altitudes, vessel for nature spirits. Weapons made from it count as magical even when unenchanted.

\vspace{5mm}

\textbf{Usages:}

\begin{itemize}[noitemsep]
\item[] Herbalism: \hyperref[Serren wood incense]{Serren wood incense}
\end{itemize}

\subsubsection{Sezarad}
\label{Sezarad}

Broad vivid flower with a short brittle root.

\vspace{5mm}

\textbf{Usages:}

\begin{itemize}[noitemsep]
\item[] Herbalism: \hyperref[Sezard gas]{Sezard gas}
\end{itemize}

\subsubsection{Shadowtop}

Massive trees native to humid climates that grow 2 ft a year, top out at 90 ft and reach a 10 ft or more diameter. Its wood is fibrous and tough, making it difficult to carving or build with, but burns hot with very little smoke (torch lasts 2 hours) and its fibres make good ropes.

\subsubsection{Silverwood}

A type of tree nurtured by elves to grow into unique forms, is free of disease and produces delicious sap that is made into famous elven mead.

\subsubsection{Sky lotus}
\label{Sky lotus}

White stemless flower native to the elemental plane of air.

\vspace{5mm}

\textbf{Usages:}

\begin{itemize}[noitemsep]
\item[] Herbalism: \hyperref[Sky lotus extract]{Sky lotus extract}
\end{itemize}

\subsubsection{Sleepweed}
\label{Sleepweed}

This plant appears similar to milkweed, and its pods contain sleep-inducing spores.

\vspace{5mm}

\textbf{Usages:}

\begin{itemize}[noitemsep]
\item[] As is: Pods can be thrown as a ranged weapon (30/60 ft range), and a struck target must make a DC 12 CON save or fall asleep for 1 minute or until it receives damage.
\item[] Herbalism: \hyperref[Sleep powder]{Sleep powder}
\item[] Poisons: \poison
\end{itemize}

\subsubsection{Slumberweed}
\label{Slumberweed}

Small rare flower with dark green leaves and purple petals.

\vspace{5mm}

\textbf{Usages:}

\begin{itemize}[noitemsep]
\item[] Herbalism: \hyperref[Tears of false death]{Tears of false death}
\end{itemize}

\subsubsection{Smuggler's Root}
\label{smugglers_root}

This illegal herb has been outlawed (hence the name). It was giving people strange abilities for limited amounts of time. This substance was used shortly for military use until they realized how addictive this stuff was. It was causing fights to erupt from within the military. It is said that ingesting one of these roots can give you wizard like powers. The trick is they are VERY hard to find.

\subsubsection{Snap Flower}
\label{snap_flower}

Orchids that act like semi- sentient Venus fly traps. These small beautiful flowers often eat small animals and can leave deep bite marks. Periodically they will shed petals lined with sharp teeth, when dried and ground up it creates a powerful anticoagulant that is used for heart medication... or on a blade to stop your opponents wounds from sealing.

\subsubsection{Sunberry}
\label{sunberry}

Found deep in dark forests, sunberries are small, round yellow berries with a faint glow. They taste sweet as honey with a mild sour afterbite. They can be distilled into a golden whisky known as sunshine which sheds light with half the intensity of a torch indefinitely. It is very expensive and tends to be served in crystal goblets that can catch and refrect the light into magnificent displays.

\subsubsection{Snowflake Lichen}
\label{Snowflake Lichen}

Magical plant that looks like snow and grows on rocks in cold climates.

\vspace{5mm}

\textbf{Usages:}

\begin{itemize}[noitemsep]
\item[] Herbalism: \hyperref[Snow oil]{Snow oil}
\end{itemize}

\subsubsection{Soarwood}

Rare wood native to higlands that possesses magical buoyancy. Water-vessels made from Soarwood are usually expensive flagships that move a lot faster then other ships. Soarwood items are 1/4 the weight of normal wooden items. It is a necessary component for the construction of airships, and when worked into an airship magically, it becomes naturally lighter than air.

\subsubsection{Spotty Dragonfire}
\label{Spotty Dragonfire}

Wildflower with red, yellow and orange petals that grows to 1 ft high and stretches 6 inch in diameter and only blooms at night. It can be found in tropical regions and grows solitary, except near red dragon lairs, where it glows plentiful.

\vspace{5mm}

\textbf{Usages:}

\begin{itemize}[noitemsep]
\item[] Herbalism: \hyperref[Dragon brew]{Dragon brew}
\end{itemize}

\subsubsection{Spark needle}
\label{spark_needle}

A kind of weed that's causes minor prickling and rashes. When grilled the flesh can be eaten, causing minor hallucinations and a tingly "static electricity" kind of feeling across the skin.

\subsubsection{Stoneshroom}
\label{Stoneshroom}

Chalky rock-looking fungus native to the elemental plane of earth that is both edible and produces spores in the form of breathable air. Stoneshroom subsists on minerals in the rock.

\vspace{5mm}

\textbf{Usages:}

\begin{itemize}[noitemsep]
\item[] Herbalism: \hyperref[Stone cookie]{Stone cookie}
\end{itemize}

\subsubsection{Stygian Pumpkin}
\label{Stygian Pumpkin}

A sulphur-scented dead-looking variety of pumpkin that can grow in any temperate region and is culitvate by goblins as food. It grows rapidly over large areas, rendering the soil poisonous to other plants.

\vspace{5mm}

\textbf{Usages:}

\begin{itemize}[noitemsep]
\item[] Herbalism: \hyperref[Devil's soap]{Devil's soap}
\end{itemize}

\subsubsection{Sunflower of Pelor}
\label{Sunflower of Pelor}

Large sunflower commonly found where undead were destroyed by Pelorian turn undead.

\vspace{5mm}

\textbf{Usages:}

\begin{itemize}[noitemsep]
\item[] Herbalism: \hyperref[Mist of Pelor]{Mist of Pelor}
\end{itemize}

\subsubsection{Sussur aka. Deeproot}
\label{Sussur}

A rare, magical tree with long gnarled branches and banyan-like aerial roots found in the largest underdark caverns. Grows to 60 ft of height, has very few leaves, and absorbs magic, creating massive (i.e. several 100 ft) antimagic fields.

\vspace{5mm}

\textbf{Usages:}

\begin{itemize}[noitemsep]
\item[] Herbalism: \hyperref[Deeproot seedling]{Deeproot seedling}
\end{itemize}

\subsubsection{Suth}

Greybark tree with long, soft, olive-green leaves that likes to grow horizontal to the ground and then double back at an angle. Suths that grow together tend to intertwine, forming wall-like barriers. Wood is hard and durable, making it difficult to work, but sheets of this wood retain strength for decades, making it great for book-covers. It also makes good shield-wood, especially since soaking it in water before battle keeps it from catching fire.

\subsubsection{Swamp tree}
\label{Swamp tree}

Dead looking tree that grows in swamps. Loved by hags and undead for its faul magics.

\vspace{5mm}

\textbf{Usages:}

\begin{itemize}[noitemsep]
\item[] Poisons: \poison
\end{itemize}

\subsubsection{Tahtoalethi (Wishfern)}

Mystal plant that grants a wish every 100 years on the night of the winter solstice.

\subsubsection{Tekkil}
\label{Tekkil}

Succulent swamp plant with fat red leaves.

\vspace{5mm}

\textbf{Usages:}

\begin{itemize}[noitemsep]
\item[] Herbalism: \hyperref[Tekkil pill]{Tekkil pill}
\end{itemize}

\subsubsection{Tereeka Root}
\label{Tereeka Root}

Slime white tuber native to the shaded sandy ground with a bitter taste.

\vspace{5mm}

\textbf{Usages:}

\begin{itemize}[noitemsep]
\item[] Herbalism: \hyperref[Tereeka meal]{Tereeka meal}
\end{itemize}

\subsubsection{Thistledown}

Plant used to make silken fabric made by elves, that can be worked into armor to make it easier to move in.

\subsubsection{Torchstalk}

Nonmagical mushroom that can serve as a torch. Burns for 24 hours without much smoke, shedding bright illumination in 10 ft radius.

\subsubsection{Trueroot}

A legendary sapling that is said to have been accidentally created during a series of experiments involving the repeated grafting of various magically enhanced roots onto treants, from which seeds were planted to create saplings that were then grafted with each other and so on. The trueroot sapling's roots are said to have pulled magical energy from an unknown place, and this magical energy could be channeled into other plants to accelerate their growth to be 10 times faster.

\subsubsection{Twilight Green}
\label{Twilight Green}

Distant belladonna relative, grows small purple poisonous berries.

\vspace{5mm}

\textbf{Usages:}

\begin{itemize}[noitemsep]
\item[] Herbalism: \hyperref[Twilight dreams]{Twilight dreams}
\end{itemize}

\subsubsection{Tyrant's sword}
\label{Tyrant's Sword}

Coarse grass with broad sharp leaves with silver edges that grows to 2 ft in hight. Sporadically found in tundras and temperate plains, it grows slow and doesn't compete well.

\vspace{5mm}

\textbf{Usages:}

\begin{itemize}[noitemsep]
\item[] Herbalism: \hyperref[Frost Lotion]{Frost Lotion}
\end{itemize}

\subsubsection{Visma}
\label{Visma}

Tropical bush with dark broad leaves.

\vspace{5mm}

\textbf{Usages:}

\begin{itemize}[noitemsep]
\item[] Herbalism: \hyperref[Visma Paste]{Visma Paste}
\end{itemize}

\subsubsection{Vodare}
\label{Vodare}

Flower that grows on graves of Rallaster worshippers.

\vspace{5mm}

\textbf{Usages:}

\begin{itemize}[noitemsep]
\item[] Herbalism: \hyperref[Vodare madness]{Vodare madness}
\end{itemize}

\subsubsection{Webcap}
\label{Webcap}

Small brownish yellow mushroom.

\vspace{5mm}

\textbf{Usages:}

\begin{itemize}[noitemsep]
\item[] Poisons: \poison
\end{itemize}

\subsubsection{Waterorb}
\label{Waterorb}

Bulbous aquatic fungus that grows in boulder-like patches in detritus areas.

\vspace{5mm}

\textbf{Usages:}

\begin{itemize}[noitemsep]
\item[] Herbalism: \hyperref[Dried waterorb]{Dried waterorb}
\end{itemize}

\subsubsection{Weirwood}

Rare oak-like tree with leaves that are a silver-sheen brown on top and velvet black on the underside, often protected by dryads and treants. Can grow huge and many-branched, will not burn from non-magical fire, and imparts a warm clear tone to musical instruments made from it. It can replace oak or holly in any spell. Weirwood within the illumiation radius of a magical light source emits light as a candle for half a minute after leaving the area.

\subsubsection{Wild Fireclover}
\label{Wild Fireclover}

Brilliant orange-red summer-flower found in temperate plains and farmland. Crushed petals give of a lovely smell for 1 week.

\vspace{5mm}

\textbf{Usages:}

\begin{itemize}[noitemsep]
\item[] Herbalism: \hyperref[Mindfire]{Mindfire}
\end{itemize}

\subsubsection{Wildwood aka. Saelas}

A flexible wood that can be worked into armor and other items. Wildwood heals a point of damage over 24 hours if exposed to sunlight for at least 1 hour or heals 5 points if also left to soak in water for 8 hours.

\subsubsection{Witchweed}
\label{Witchweed}

A plants whose leaves and stalks are purple and green.

\vspace{5mm}

\textbf{Usages:}

\begin{itemize}[noitemsep]
\item[] Herbalism: \hyperref[Smokestick]{Smokestick}
\end{itemize}

\subsubsection{Wittlewort}
\label{Wittlewort}

Herb with green gossamer-like fronds which, due to its rapid growth cycle, is found onliny in the spring in temperate tropical areas. Deters slugs and other pests.

\vspace{5mm}

\textbf{Usages:}

\begin{itemize}[noitemsep]
\item[] Herbalism: \hyperref[Wittlewort brew]{Wittlewort brew}
\end{itemize}

\subsubsection{Wolfsbane aka. Belladonna}
\label{Wolfsbane}

Wolfsbane is small brush with black, highly toxic berries.

\vspace{5mm}

\textbf{Usages:}

\begin{itemize}[noitemsep]
\item[] Herbalism: \hyperref[Wolfbane spring]{Wolfbane spring}
\item[] Poisons: \poison\poison
\end{itemize}

\subsubsection{Wolves' Milk}

Flower that contains white milky liquid. Commonly found in fields and grasslands.

\subsubsection{Wolfweed}
\label{Wolfweed}

Similar in appearance to Wolfsbane.

\vspace{5mm}

\textbf{Usages:}

\begin{itemize}[noitemsep]
\item[] Herbalism: \hyperref[Wolfbane spring]{Wolfbane spring}
\end{itemize}

\subsubsection{Yarpick aka. Daggerthorn}

A type of tree that grows small fruit whoe seeds are nourishing both whole and as ground meal.

\subsubsection{Zephyr Bloom}
\label{zephyr_bloom}

A flower commonly found in the foothills of mountains. The light blue petals grow in the shape of a tornado. If the petals are used to make a tea, the imbiber feels like they are flying for about an hour.

\subsubsection{Zurkhwood}

Giant 30-40 ft high mushroom. Has large spores that can be eaten if prepared properly, and its hardy stalks serve as an underdark substitute for wood.
